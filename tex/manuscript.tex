\begin{article}
\hypertarget{uxc9c4uxc220uxc11c}{%
\chapter{진술서}\label{uxc9c4uxc220uxc11c}}

한국에서 병역의 의무는 국민의 의무로서, 다시 말하면 국민과 비국민을 검증해내는 중요한 장치로서 기능합니다. 이때 병역의 의무를 이행하지 않겠다고 선언하는 것은 이 땅에서 비국민의 자리에 서겠다는 선언이기도 합니다.

제가 병역을 거부하는 이유는 아주 단순합니다. 군대에 가고 싶지 않기 때문입니다. 군인이 되고 싶지 않기 때문입니다. 군사의 경험으로 검증 받아 이 나라에 합류하고 싶지 않기 때문입니다.

조직화된 폭력은, 특히 군대가 행사하는 폭력은 의견을 가져선 안됩니다. 군대가 하는 일은 국가의 시민을 보호하는 것이고, 이 폭력은 절대 정당해야 할 것입니다. 이 폭력은 절대 정당할 것이기 때문에 사실 의견을 가질 필요가 없습니다. 이러한 폭력의 정당성을 획득하는 사회구조적 장치가 없다면 이 나라는 건전하게 운영될 수 없을 것입니다.

이때 또 다시 대체 어떤 폭력이 정당할 수 있는지에 대한 의문이 생깁니다. 위에 열거된 문제들은 조직화된 폭력이 정당성을 획득하는 방식에 대한 문제일 뿐 정작 그 폭력이 정당하느냐는 질문에 대한 답은 아니기 때문입니다. 만약 포괄적으로 폭력이 정당하다고 한다면 한국의 군사체계는 물론이고 사법체계의 일부는 다시 구축되어야 할 것입니다. 만약 `어떤', 특수한 폭력이 정당하다고 한다면 정당하지 않은 폭력이 무엇인지 질문해야 할 것입니다. 얼마 안 있어 우리는 부당한 폭력과 정당한 폭력의 경계에 설 것입니다. 그때 다시 한 번 잊었던 질문에 사로잡힐 것입니다. `어떤 폭력이 정당한가?' 라는 질문 말입니다.

저는 폭력의 정당성을 판단하는 구조에 몸을 맡기는 대신 이 판단의 중심에 서서 조직화된 폭력에 합류하기를 거부할 것입니다. 또한 이 거부에 대해 요구되는 방식에 따라 책임질 것입니다.

한국의 군대는 군인에게 마땅히 싸워야 할 적에 대해 교육합니다. 하지만 저는 그러한 교육에 저항합니다. 그 적이 이미 결정되어 있고, 그 결정을 전제로 한 명령을 받는다면 저는 그 명령에 대해 거부할 것입니다. 군대에서 생산되는 모든 명령은 적의 존재에 근거하고 있습니다.

하지만 적이 누구인지는 제가 결정할 문제입니다. 그 누구도 이 신중한 결정체계에 끼어들어선 안될 것입니다. 또한 이에 대해서는 오롯이 제가 책임져야 할 것입니다. 이 말은 제가 국가가 결정한 주적의 개념에 대해, 실체에 대해, 제가 이를 받아들일 것인지에 대한 판단을 유보하겠다는 것입니다. 이 판단 유보에 대해서도 오롯이 제가 책임져야 할 것입니다.

다시 한 번 말씀 드립니다. 본인, 김승해는 병역을 거부합니다. 이유는 군대에 가고 싶지 않기 때문입니다. 군인이 되고 싶지 않기 때문입니다. 군사의 경험으로 검증 받아 이 나라에 합류하고 싶지 않기 때문입니다.
\end{article}

\begin{article}
\hypertarget{uxbcf5uxb3c4-uxcc9cuxc0acuxb4e4}{%
\chapter{복도, 천사들}\label{uxbcf5uxb3c4-uxcc9cuxc0acuxb4e4}}

목매단 말들이 기어이 기어나가\\
문간에 엎드렸다 잘린 머리통이 온 복도를\\
굴러다녔다. 그 소리가 새벽을 소란스레 찢어 놓았다\\
매달린 새벽에 우린 숨을 멈추었다\\
오늘 새벽 우린 숨없이 잠들었다 아침\\
을 기약하지 않기로 하였다\\
잘린 목에서 피거품이 일었다\\
천사는 그 위를 걸어 지나가다 넘어지고 말았다\\
잘린 머리통이 누운 천사의 어깨 위로 올라탔다\\
천사가 진저리를 치며 일어났다

우린 우려한다\\
복도는 천사의 머리칼을 흔들 것이다\\
흔들린 머리칼이 복도의 유리창을 세게 긁을 것이다\\
우린 잘리다 만 귓불을 애써 붙이며\\
목매단 해명을 늘어놓을 것이다\\
목매단 말들이 기어이 기어나가 문간에 엎드렸다.
\end{article}

\begin{article}
\hypertarget{uxac70uxc2e4}{%
\chapter{거실}\label{uxac70uxc2e4}}

거실에는 화장실이 하나씩 있었다\\
각 거실의 수채 구멍은 하나였고 단\\
하나의 수채 구멍만이 물을 바깥으로 배출\\
했다 이때 물과 잔반과 똥\\
오줌은 우리 죄의 일부가 아님이 확실해진다\\
우린 거실에서 항상 아가리를 놀려야만 했다\\
입을 열고 헛소리, 그 헛소리 엉뚱한\\
말들을 하다 보면 말들은 아랫입술을 축이고\\
거실을 가득 채웠다 우리가 말없이 머물 때\\
거실을 채우고 있던 시간이 점점 수채구멍으로\\
밀려 내려왔고 시간이 없을 때\\
열린 창문 틈으로 말이 흘러 내려갈 때\\
복도가 소리 없이 침묵으로 가득 찰 때\\
티브이가 켜졌다\\
드라마가 시작되었다\\
그때 우리는 말없이 있을 수 있었다
\end{article}

\begin{article}
\hypertarget{uxc728uxbc95}{%
\chapter{율법}\label{uxc728uxbc95}}

율법이 작성될 때마다\\
문이 열리고 천사가 방에 들어온다\\
천사는 작성된 율법을 수거한다\\
율법은 율법이 작성되어선 안된다는\\
계율을 포함한다 율법이 작성될\\
때야 말로 이단적 수도승들이 천사와\\
마주칠 수 있는 몇 안되는 기회다\\
율법의 각 항목은 매일 아침 망가진 스피커에\\
의해 포고된다. 우리 중 아무도 그 포고에\\
귀 기울인 사람은 없었고 그것이 가능 하지도\\
않았다 그저 귀를 막고 신경질을 내는\\
것 말고는 할 수 있는 일이 없었다
\end{article}

\begin{article}
\hypertarget{uxc0b0uxc758-uxc911uxd131uxc5d0uxc11c}{%
\chapter{산의 중턱에서}\label{uxc0b0uxc758-uxc911uxd131uxc5d0uxc11c}}

산 중턱에 도착한 그는 시야가 어두워지기 시작했음을 깨달았다. 그들도 알지 못한 사이에 숲의 가장 깊은 곳에 도달한 것 같았다. 산의 비탈을 따라 내려오던 그들이었다. 그 동안 희미하게나마 길을 밝히던 달빛도 감지되지 않았다. 하늘을 보려 고개를 올렸지만 달라 보이는 것은 아무것도 없었다. 가끔 그들의 눈 앞을 지나가던 풀벌레들이 있었다. 달빛을 가로지르며 지나갈 때마다 수면이 깨어지듯 빛이 일었다. 하지만 지금은 아무것도 보이지 않았다. 곧 다시 밝아질 것이다. 이 모든 것들이 지나갈 것이다. 이런 무력한 낙관으로 당혹감을 떨쳐내야만 했다. 몇 걸음 걸어 보았지만 곧 멈추어야만 했다. 한 발 내딛는 일이 마치 깊은 골짜기를 뛰어 건너는 일처럼 느껴졌기 때문이다.

아무것도 보이지 않았다. 그들은 온 세상이 어둠에 휩싸였다고 확신했다. 드디어 어둠이 그들을 찾아왔다고 생각했다. 이곳이 산의 중턱이었다.

온 세상이 어둠에 휩싸이기 전이었다. 산에서 내려오다 마주친 중년 남성이 그들에게 말했다:

아무것도 보이지 않는 순간이 올 것이다. 산의 중턱에서 그 순간에 직면할 것이다. 다들 그 때를 준비해야만 한다. 그 때가 언제인지 알 수 없다. 산의 중턱이 어디서부터 시작할 것인지 알 수 없다. 산의 중턱이 어디서 끝나는지 알 수 없다. 나는 아무것도 모른다. 당신들에게 내가 말할 수 있는 단 한가지 진실을 일러주겠다. 나는 이미 산의 중턱에 서 있다. 지금 이곳에서 빛을 기다리고 있다. 당신들은 아직 산의 중턱에 도달하지 않은 모양이다.

이들이 다시 걷기로 결정한 것은 한참 뒤의 일이었다. 그들은 일어나 어떻게든 계곡 사이를 뛰어넘어 가기로 했다. 빛은 오지 않을 것이다. 이들의 걸음은 산의 중턱에 도착한 순간부터 도약의 연속이 되었다. 몇 번의 도약이 있고, 그들 중 하나가 넘어졌다. 팔꿈치가 쓰리지만 벌떡 일어나 다시 걷기 시작했다. 둘 중 하나가 크게 다쳤고 얼마간 쉰 뒤 부축을 받으며 걸었다. 둘 중 하나가 넘어지고 앞니가 박살 났다. 턱이 두 동강났다. 어깨가 빠졌다. 무릎이 쓸렸다. 발목이 꺾였다. 별 수 없었다. 어둠 속에서 그들이 할 수 있는 일은 그럼에도 하던 일을 계속 하는 것이었다. 멈추어 선 상태에선 아무런 가능성도 그들에게 찾아오지 않는다. 그들은 이런 어둠을 준비해본 적도 없었고, 그에 대한 준비가 가능하긴 한 건가. 다만 걷는 것 말고는 방법이 없어 보였다.

아픔을 동여 메며 바닥에 나뒹굴고 있던 그들을 한 사람이 지나쳐 갔다. 그는 바닥에 긴 불꽃을 그리며 지나가고 있었다. 불꽃에 드러난 것으로 보면 그는 마치 머리를 끌며 기어가는 것처럼 보였다. 그를 불러 어떻게 하면 불꽃을 그릴 수 있는지 물어보고자 했으나 그는 멈추지 않았다. 그가 지나간 자리에서 새빨간 흙이 꺼져가고 있었다. 그가 그린 궤적을 따라 내려오는 한 무리가 있었다. 무리 역시 모른 채 지나갔다. 붉은 흙이 사그라들었다. 그들은 다시 어둠 한 가운데 걷는 신세가 되었다.

불꽃의 까닭을 알게 된 것은 그로부터 한참이 지나서였다. 그들 중 하나가 쓰러지며 바닥에 얼굴을 심하게 다쳤는데 그 과정에서 두개골의 일부가 드러나게 되었다. 다른 하나가 두개골이 땅에 부딪치며 발생한 불꽃을 보게 되었다. 이렇게 불꽃의 원인이 규명되었다. 그들은 애써 머리 가죽을 벗겨내고 땅에 머리를 부딪쳤다. 나머지 하나는 그 불꽃으로 앞으로 갈 길을 가늠하는 것이다. 이전에 본 사람처럼 계속해서 머리를 끌고 길을 갈 순 없을 것이라고 생각했다. 정작 머리를 끄는 본인은 앞을 볼 수 없었기 때문이다.

이내 이러한 문제점 역시 해결되었다. 둘 중 하나가 머리를 끄는 동안 생기는 빛으로 나머지 하나가 앞을 판독하면 된다는 아이디어가 있었고, 그것이 실제로 가능함을 알게 되었기 때문이다. 그럼에도 어쨌든 이런 식으로 앞으로 가기 위해서는, 즉 빛을 만들기 위해서는 두개골이 드러나야만 했고, 그것은 당사자에게 무척 고통스러운 일이었다. 다만 그 고통이 이제 한 사람에게 집중된다는 점에서 이전에 그들이 취한 방식보다 진보했다고 볼 수 있었다. 모두가 아플 수는 없는 일이었다. 머리를 끄는 동안 하나는 비명을 참기 위해 애써야만 했고, 나머지 하나는 억눌린 비명을 고스란히 들으며 그의 시야를 겹겹이 메운 의심을 몰아내야만 했다. 이러한 구도는 어떻게든 고착될 수 밖에 없었다. 그것이 안쓰러워 나머지 하나가 두개골을 드러내고 땅을 기며 피고름에 시야를 가릴 순 없었기 때문이다.

이런 식의 여정이 이어지고, 놀라운 일이 발생하는 것을 관찰할 수 있었다. 그의 두개골이 발달하기 시작한 것이다. 이미 그의 머리에서 안구는 곪아 사라졌고, 그 사이를 응고된 핏덩어리가 가득 차 있었다. 코 또한 쓸려 사라졌다. 목뼈는 이미 뒤로 휘어 있었으며 비명을 감추기 위해 부푼 혀가 입안을 채우고 있었다. 터진 고막이 펄럭이는 그의 귀는 이제 호흡기 역할을 하고 있었다. 몇 번이나 사지를 헤매고, 으깨어진 뇌가 귀에서 흘러나오는 바람에 빛이 빠르게 사그라진 것이 몇 번이었다. 더 이상 97 서로의 역할을 바꿀 수 없을 만큼 멀리 갔기에, 몹쓸 동정심으로 이 여정을 포기하기로 결정하려던 참이었다. 그런데 두개골이 빠른 속도로 발달하며 머리를 키우기 시작한 것이다! 이것은 그들 중 나머지 하나의 진술이었다. 합리적인 의심으로 그 진술을 다시 들여다보면 그것은 뼈보다는 아주 단단한 각질에 가까웠을 것이라고 생각한다. 지금 이곳을 낮게 울리는 거대한 존재들의 윤곽을 살펴보건대 그것은 여기 이들의 그것과 동일한 것일 것이다.

이따금 그들을 지나치던 무리들이 있었다. 대체로 세 명에서 네 명으로 이루어져 있었다. 그들은 대부분 한 명의 `머리통'을 데리고 있고 또 하나의 교대할 머리통을, 나머지는 반쯤 기능을 잃은 눈으로 길을 판독하는 역할을 했다. 이들은 아주 빠른 속도로 길을 찾아 나갔는데, 그러기 위해서 머리통이 탈진하기 직전까지 몰아붙인다. 결국 그 머리통이 탈진하고 나면 나머지 머리통이 뒤를 잊고 탈진한 머리통을 한 명이 부축해 가는 방식이었다. 저들이 두 머리통을 어떻게 무리에 포섭했는지, 혹은 무리 내에서 뽑았는지는 약간만 상상해보아도 아주 의심스러웠으나, `그들' 또한 한 명의 고착된 머리통 역할을 하는 사람이 있다는 사실과 그것이 주는 죄책감 때문에 감히 물어보지 않았다. 이런 일들은 항상 아무 말 없이 이루어지는 법이다.

또 다른 무리는 탈진한 머리통을 대하는 방식에 있어서 이전에 서술한 무리와 좀 다른 방식을 취했다. 다른 누군가가 탈진한 머리통을 끌고 가는 것이다. 이 방법은 무리 안에서 머리통을 하나만 보유하면 된다는 장점이 있으나 그 머리통을 끌고 가야 할 사람이 탈진할 경우를 대비해 나머지 하나가 길을 살피는 한편 교대를 위해 준비하는 방식이었다. 이 방식은 이전의 무리에 비해 무리의 덩치를 더 줄일 수 있다는 장점이 있었다. 머리통은 거의 가사상태로 내버려둔다. 역시 이 무리의 머리통이 어디서 왔는가에 대한 질문에 떳떳하게 대답하기 힘든 부분이 있었고, 끌고 가는 사람이 앞에 위치해야 했기 때문에 낭떠러지에 떨어질 위험에 무방비로 노출되어 있다는 문제가 있었다.

어떤 무리가 다른 무리와 마주치게 될 경우에는 아무래도 난감한 상황이 벌어진다. 각 무리간에 싸움 전야의 긴장감이 돌기 시작한다. 그 무리가 몇이건 상관 없다. 어떤 무리가 먼저 앞으로 가면 생기는 불꽃으로 나머지 무리가 뒤따라간다. 물론 나머지의 머리통은 땅에서 떨어진 상태로. 아무리 굳은살이 빠르게 발달한다고 해도 그것은 어떤 국면을 넘어서야만 한다. 대개의 머리통은 그것을 못 버티고 죽는다. 대부분의 무리가 그런 머리통을 보유한 상태가 아니었다. (보유다. 나는 분명히 보유라고 썼다. 나는 이런 표현을 사용하는 내 아가리를 두 손으로 확 찢어버리고 싶다.) 때문에 그 국면 이전에 있는 머리통을 땅에 끄는 것은 무리에 있어서 굉장한 손실일 수밖에 없다.

이런 신경전 끝에 여러 무리가 한 무리로 통합되는 경우가 있고, 치열한 싸움 끝에 대부분이 죽는 일도 벌어진다. 이럴 때는 살아남은 몇 사람들의 죽음의 긴장을 안고 한 무리로 통합한 후 다시 산 아래로 향한다.

나 또한 언젠가 머뭇거리는 무리들과 마주친 일이 있다. 나는 아직 산의 중턱에 도착하지 않았기 때문에 그들의 길을 안내하길 자청했다. 그들 또한 내가 그곳에 도착하지 않은 것을 눈치채고 무리에 끌어들이고자 노력했으나 나는 거절했다. 다만 그 때의 내 걸음은 긴장의 연속이었는데 그것은 이미 산의 중턱에 대한 말을 들었고, 이렇게 걸음을 재촉하다 산의 중턱에 이르게 될 것이라는 불안감 때문이었다.

무리들 사이에는 희미한 빛을 덮으며 등장하는 산 너머의, 산 아래의, 산의 골의 엄청난 빛의 근원에 대한 소문이 돌고 있었다. 저 빛은 너무나 강해 이들이 길을 찾는데 전혀 도움이 되지 않았고, 산의 중턱에 닿기 전 그 빛의 근원을 본 사람들은 불안에 떨며 그것이 거대한 머리였다고, 거대한 바위였다고, 거대한 산이었다고, 거대한 산이 움직이며 우리가 산 아래 도달하는 것을 방해하고 있다고 말했다. 또한 그 거대한 것이 이동할 때 울리는 대지의 소리는 이미 공포 그 자체였다. 그들 중 하나는 누군가로부터 들은 이야기를 했다. 운동능력을 갖춘 머리통이 스스로 팔과 다리를 움직여 무리 그 누구도 보지 못한 거대한 머리를 끌며 빠르게 이동하는 것을 보았다고. 그 빛이 다른 머리통에 비해 강했기에 그가 지나간 땅은 한동안 달아올라 있었고, 그 빛을 따라 이동한 경험이 있었다는 것이다. 그러며 그런 식으로 산 아래로 내려가다 보면 거대한 머리통이 될 수도 있을 것이라고 말했다.

이들 무리는 오랫동안 나의 지시에 따라 움직였고, 이후에 나는 그들과 결별하게 되었다. 이들은 이미 각자 머리통을 갖고 있었고 머리통들이 회복되기 시작하면서 피부가 두개골을 가리기 시작했기 때문이다.

나는 얼마 후 내게 굳은살에 대해 말한 `그들'을 또 보게 되었다. 그들로부터 내가 이전에 안내한 것으로 추정되는 무리들에 대한 소식을 듣게 되었다. 둘이 머리를 끌며 이동하던 중 십 수명의 시체와 마주치게 되었는데 마침 발생한 산 아래의 빛을 통해 그들 중 몇 명의 머리통에서 각질이 발생하기 시작했다고 한다. 하지만 그들의 머리통은 이미 너덜너덜하게 박살 난 상태였다고 한다. 추정컨대 그들은 각질이 생긴 머리통의 독점여부를 놓고 다툼을 벌인 것으로 보인다. 다만 머리통의 시체에서 발생한 각질은 여전히 쓸모 있을 것이라고 생각했고, 그들은 그의 머리통의 머리에 굳은 살이 생긴 머리통의 일부를 보철하는 시도를 해 보았다고 한다. 그 시도는 효과가 있었다. 이미 죽은 머리통은 꽤 큰 상태로 발달했고, 보철한 머리통 세 개 중 하나로부터 굳은 살이 발달하기 시작해 그와 함께하던 머리통의 각질과 더불어 함께 자라기 시작했다고 한다. 나는 그것과 상관 없이 기꺼이 그들을 안내하기로 하였는데, 그것은 처음 내게 중요한 정보를 선뜻 알려준 이들에 대한 예우였다.

우린 오랫동안 길을 걸었고, 나의 불안이 완전히 가신 것은 아니지만, 그럼에도 우리 사이에는 일종의 우정이라고 할 만한 감정이 오갔다. 그가 보인 머리통에 대한 우정, 피할 수 없는 희생에 대한 죄책감에 나는 깊이 공감했고, 그가 언뜻 내보이는 다른 가능성에 대한 모색은 내게도 깊은 충만감을 선사했다.

그러다 문제가 생겼다. 나의 실수였다. 나의 잘못된 안내로 그의 머리통이 길에 튀어나온 바위를 무리하게 넘어가려다 죽음에 이르게 되었다.

그는 친구의 죽음에 비통해 하는 한편 내게 위로를 건냈다. 나는 죽음의 슬픔보다 실수에 대한 책임에 사로잡혀 겁에 질려 있을 뿐이었다. 그는 서럽게 울다 나를 위로했다. 나는 위로 한 마디 못하고 그저 그 상황에 있어서 내 잘못이 무엇이었는지, 누구의 잘못이었는지 판단하는 일에 사로잡혔다. 결국 나는 이 모든 것이 전적으로 나의 잘못이라고 생각하고 거대한 책임에 사로잡혀 길에 주저앉았다.

얼마간 시간이 지나고 죽은 머리통의 경련이 멈추었다. 그가 일어나 내게 말했다. 내게 큰 돌을 구해다 주렴. 한 무리가 우릴 지나치며 혀를 찼다. 나는 일어나 큰 돌을 찾아 그에게 건냈다. 그는 그 돌을 죽은 머리통의 뒤통수에 내려찍기 시작했다. 뒤통수가 내려앉고 뇌수가 튀었다. 그는 완전히 내려앉은 머리통의 머리를 완전히 열고 그 안에 있는 기관들을 끄집어내기 시작했다. 그가 손으로 끄집어낼 수 있는 것은 모조리 들어낸 후 그 안에 가득 찬 피를 빨아내 밖으로 뱉어냈다. 이후 그 안에 흙을 넣어 내부를 깨끗이 닦고 흙을 모두 꺼냈다. 그가 이 모든 일을 끝내고 숨을 돌리며 머리통의 머리맡에 앉았을 때 이를 부딪치며 떠는 내게 말했다. 별 수 없었다.

그리고 그는 일어나 텅 빈 머리통의 머리에 자신의 머리를 집어넣었다. 그리고 내게 인사를 남기고 단단하게 굳은 시체를 끌고 기어가기 시작했다. 내가 기꺼이 길을 안내하겠다고 소리쳤으나 그는 빠르게 내 시야에서 벗어났다. 피투성이가 된 길가에서 단 한 걸음도 옮기지 못하고 많은 시간이 지났다. 몇 개의 무리가 나를 지나쳤다. 그러다 내가 다시 일어나 길을 걷기 시작한 것은 어쨌든 별 수 없이 어떻게든 산 아래로 내려가야 한다는 판단 때문이었다. 어둠의 정오에서 나는 산의 골을 따라 움직이는 거대한 빛을 보았다. 산의 귀퉁이를 무너뜨리며 이동하는 거대한 덩어리들을 보았다. 덩어리를 따라 대지가 으르렁거리며 뒤따라갔다. 먼 산의 탄내가 내게 도달할 무렵, 나는 내가 산의 중턱에 도착했음을 깨달았다. 겁에 질려 내달리다 땅에 고꾸라져 머리를 처박은 내게 무리가 찾아왔다. 그들이 나를 끌기 시작했다.
\end{article}

\begin{article}
\hypertarget{uxce68uxbab0uxd558uxb294uxb545}{%
\chapter{침몰하는땅}\label{uxce68uxbab0uxd558uxb294uxb545}}

그때 이 땅이 침몰하기 시작했다\\
우리가 그 침몰을 감지했을 때\\
침몰은 멈추었다 그것은 더 이상\\
침몰할 것이 남아있지 않았기 때문이다\\
만약 당시 누구라도 그 곳에 알량한\\
돌 하나라도 쌓았다면 침몰은 계속되었을\\
지 모른다\\
우린 내달렸다\\
가장 높은 곳으로 우린 서로의 두\\
어깨가 찢어지도록 붙잡고 늘어졌다 우린 둘 다\\
살아남을 수는 없다 그렇다면 둘 중\\
하나는 살아남아야 하고 그것은 바로\\
나여야만 한다. 우린 서로의 생의\\
찌꺼기를 안고 살아갈 자신이 없었다

세상 언저리에서\\
벼랑 끝보다는, 간신히 안락한 곳에서\\
죽음이 보이는 곳에서\\
등지고 서서\\
나보다 이 세상이 먼저 탕진하길\\
기다리면서\\
가늘어 빠진 팔뚝에서\\
온 생이 소진되는 것을 지켜보면서
\end{article}

\begin{article}
\hypertarget{uxc6b0uxb9acuxc758-uxb05d}{%
\chapter{우리의 끝}\label{uxc6b0uxb9acuxc758-uxb05d}}

우린 끝을 말했지\\
삶의 끝은 없었지\\
다만 말할 순 있었지\\
말은 늘 무력했지

너의 머린 길었지\\
나의 목은 바닥에 닿았지\\
너는 나를 보았지\\
너는 늘 무력했지

시간이 멈췄지 말없이\\
의견은 없었지 우린 무력했지

자\\
이제 시간이 멈추고 너의 하늘이 빨갛게 무너진다 자 이제\\
시간이 멈추고 너의 하늘이 빨갛게 무너진다

밤의 길이 끝날 무렵 시간이 멈췄지\\
별은 지평선 너머로 쏟아져 내렸지

너는 지평선 너머의 소식을 말했지\\
너의 말은 저 별처럼 쏟아져 내렸지

시간이 멈췄지 말없이\\
의견은 없었지 우린 무력했지

자 이제 시간이 멈추고 너의 하늘이 빨갛게 무너진다\\
자 이제 시간이 멈추고 너의 하늘이 빨갛게 무너진다

우리가 나눈 모든 계획과\\
우리가 외면한 모든 미래와\\
우리를 지나친 모든 가능성과\\
우리가 내버린 모든 호의와

우리가 놓친 모든 약속과\\
우리를 배반한 엄숙함과\\
우리의 선한 의지와\\
그 밖에 모든 시도들을 둘러싼\\
시간이 멈추고 너의 하늘이 무너진다
\end{article}

\begin{article}
\hypertarget{uxb144-3uxc6d4-2uxc77c}{%
\chapter{2013년 3월 2일}\label{uxb144-3uxc6d4-2uxc77c}}

오전 10시 쯤 나오니 아버지와 동생, 준석이 기다리고 있었다. 아버지가 뭐가 먹고 싶느냐 물어보니 딱히 먹고 싶었던 건 없었고 둘러보니 정말 아무것도 없는 곳이었다. 근처에 순대국밥집이 눈에 띄어 그리로 가자 말했다. 국밥을 다 먹고 가쟈온 짐을 아버지께 맡기고 나는 준석과 걸어서 집으로 가보겠다 말했다.

미처 집으로 보내지 못한 몇 권의 책과 노트, 편지가 내 짐의 전부였을 것이다.

전날엔 용산참사로 들어온 김창수씨가 사면되었다. 우린 이 모든 것을 소문으로 들어 알고 있었다. 다시 보잔 이야기는 하지 않는다. 다시 보게되면 어디서 볼 지 뻔하니까. 그래도 우린 그게 사실이든 아니든 이번이 생에 마지막으로 들른 감방이라고 믿으니까.

왜인지 나는 판결이 나오고 집으로 돌아올 수 있을 줄 알았다. 법원에 들르기 전 걸어서 15분 거리에 있는 연향도서관에 들러 책을 빌려 잠시 살펴보며 바스를 타고 법원으로 갔다. 그리고 법정구속을 선고받고 바로 교도소로 들어가바렸다.

왜인지 나는 내가 집으로 돌아올 거라 생각했던 모양이다. 여호와의 증인들은 많이들 그랬으니. 법원에서 선고 받고 판결문이 나오면 검사에 의해 겨도소로 들어간다고, 나는 그렇게 듣고 이상하게도 그걸 믿고 있었다.

법정구속의 이유는 도주우려. 판사님 내가 왜 도주를 합니까? 소리를 지르며 질질 끌려간 이후 몇 주간 난 수소문해 출소하면 그 판사의 집에 불을 지르려고 별렀다. 반드시 죽여버리겠다고 생각하다 얼마 안있어 아무래도 상관 없다는 생각으로 그 마음을 접었다. 그래도 이렇게 구속될 것이었다면 도주를 한 번 해볼 걸 그랬다.

재판장 앞에 딸린 방에 들어가 포박되고 교도소에 오기까지 몇 시간 동안 구치소에 있다 교도소로 들어간게 아마 그날 오후 5시였을 것이다. 각각의 장소는 이송 버스를 타고 움직였다. 애초에 교도소가 어딨는지도 몰랐고 나는 이런 지리 정보에 대한 감각이 없는 것과 마찬가지여서 가보지 않은 곳은 잘 알지 못한다. 그러니 법원에서 교도소에 이르는 경로를 도저히 알 수 없었다.

풍경은 별로 바뀌지 않았고, 나는 근육 뚱땡이가 되었다.

형량에서 3개월 일찍 출소한 1년 3개월. 그 짧은 시간 동안 나의 어딘가가 망가졌다는 걸 깨닫는데는 더 오랜 시간이 걸렸다. 그게 완전히 망가져 이제 돌이킬 수 없다는 걸 깨닫는데는 훨씬 더 긴 시간이 필요했다. 그것이 적출 불가한 기관이라는 걸 깨닫는데는 그렇게 오래 걸리지 않았다. 정확히 나의 어디가 완전히 망가졌는지 알아차릴 때까진 더 많은 시간이 필요할 것이다.

어떤 생각은 스스로 동력을 갖고 전개되는 것 같다. 내가 약해 빠진 탓이라고, 내가 약하지 않았다면 난 더 멀쩡하고 더 나은 사람일 수 있을거라고, 이렇게 망가지진 않았을거란 생각의 작동이 멈춘 건 얼마 전의 일이었다.

교도소에서 왕지동의 법원으로, 법원에서 그날 했던 것처럼 담배를 피우며 자판기 커피를 뽑아먹었고 연향도서관으로. 이렇게 집으로 오는데 걸린 시간은 한 시간 반이었다. 이 귀가길을 위해 1년 3개월이 걸렸다.

이 책에 포함된 6편의 글 중 병역거부 진술서와 이 글을 제외한 나머지는 모두 교도소에서 작성되었다. 각각의 글은 짧고 의뭉스러울 것이다. 불시에 들어닥쳐 노트를 펼쳐보는 무식하고 야만적인 독자들의 눈길을 피하기 위함이었는데 지금 생각해보면 다 쓸데없는 짓이었을지도 모르겠다. 그냥 안쓰고 그만두면 되는 일이었다. 그게 가장 쉬운 일이었을 것이다. 내겐 이런 걸 재빠르게 알아채 수행하는 영민함이 부족하다. 게다가 그때 쓴 글들을 지금 다시 보니 누가 보아도 수상쩍어했을 것이다. 차라리 정직하게 존나 탈옥하고 싶다고 쓸 걸 그랬다.

이 책에 실릴 글들은 나의 수감 생활에 대해 아무것도 이야기해주지 않을 것이다. 한편 내가 늘 말해온 안정되고 안락했던 나의 수감 생활을 증언해줄 것이다. 안정과 안락함은 그때도, 지금도 내가 설정할 수 있는 최고의 목표니까 말이다.

당신들의 안정을 빈다. 만약 내가 방해가 되었다면, 내 알 바는 아니지.
\end{article}
