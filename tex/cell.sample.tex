%!TEX TS-program = pdflatex
\documentclass[12pt, b6paper, openany]{memoir}

    \usepackage[cm]{fullpage}
    \setstocksize{182mm}{128mm}
    \usepackage{enumitem}
    \usepackage[paperwidth=128mm, paperheight=182mm, top=1.5cm, bottom=2.2cm, inner=2cm, outer=2.5cm]{geometry}
    \usepackage{titlesec}
    \usepackage{kotex}
    \usepackage[breaklinks=true]{hyperref}
    \hypersetup{colorlinks, citecolor=black, filecolor=black, linkcolor=black, urlcolor=black}
    
    \usepackage{epigraph}
    \setlength\epigraphwidth{1\textwidth}
    
    \makeatletter
    \newcommand*{\cleartoleftpage}{\clearpage\if@twoside\ifodd\c@page\hbox{}\newpage\if@twocolumn\hbox{}\newpage\fi\fi\fi}
    \makeatother
    
    \makeatletter
    \newcommand*{\cleartorightpage}{\clearpage\if@twoside\ifeven\c@page\hbox{}\newpage\if@twocolumn\hbox{}\newpage\fi\fi\fi}
    \makeatother
    
    \usepackage{titling}
    %define titlingpage
    \pretitle{\begin{flushleft}\begin{normalsize}\begin{textbf}}
    \posttitle{,}
    \preauthor{\end{textbf}}
    \postauthor{,}
    \predate{}
    \postdate{\end{normalsize}\end{flushleft}}
    % Set title of tableofcontent
    \renewcommand{\contentsname}{차례}
    % set tableofcontent
    \maxtocdepth{chapter}
    \renewcommand{\baselinestretch}{1.355                                                                                                                                                                                                 }
    \setsecnumdepth{part}
    \setlength{\beforechapskip}{0pt}
    
    \titleformat{\chapter}{\filright}{}{0pt}{\normalfont\large\bfseries}
    \titlespacing*{\chapter}{0pt}{0pt}{2\baselineskip}
    
    \setlength{\parskip}{1em}
    
    \newenvironment{lastnote}{%
        \clearpage\vspace*{\fill}%
        \begin{footnotesize}
    }{%
        \end{footnotesize}
    }
    
    \newenvironment{lyric}{\setlength{\parindent}{0pt}}{}
    \newenvironment{article}{}{}
    
    
    
    
    \setcounter{secnumdepth}{0}
    
    % Footnotes: 
    
    
    % Title, authors, date.
    \title{1x4 샘플북}
    \author{김승해}
    \date{2018}
    
    \begin{document}
    
    \frontmatter
    \begin{titlingpage}
    \maketitle
    \end{titlingpage}
      \tableofcontents
      
    \mainmatter
    \begin{article}
    \hypertarget{uxc9c4uxc220uxc11c}{%
    \chapter{진술서}\label{uxc9c4uxc220uxc11c}}
    
    한국에서 병역의 의무는 국민의 의무로서, 다시 말하면 국민과 비국민을 검증해내는 중요한 장치로서 기능합니다. 이때 병역의 의무를 이행하지 않겠다고 선언하는 것은 이 땅에서 비국민의 자리에 서겠다는 선언이기도 합니다.
    
    제가 병역을 거부하는 이유는 아주 단순합니다. 군대에 가고 싶지 않기 때문입니다. 군인이 되고 싶지 않기 때문입니다. 군사의 경험으로 검증받아 이 나라에 합류하고 싶지 않기 때문입니다.
    
    조직화된 폭력은, 특히 군대가 행사하는 폭력은 의견을 가져선 안 됩니다. 군대가 하는 일은 국가의 시민을 보호하는 것이고, 이 폭력은 절대 정당해야 할 것입니다. 이 폭력은 절대 정당할 것이기 때문에 사실 의견을 가질 필요가 없습니다. 이러한 폭력의 정당성을 획득하는 사회 구조적 장치가 없다면 이 나라는 건전하게 운영될 수 없을 것입니다.
    
    이때 또다시 대체 어떤 폭력이 정당할 수 있는지에 대한 의문이 생깁니다. 위에 열거된 문제들은 조직화된 폭력이 정당성을 획득하는 방식에 대한 문제일 뿐 정작 그 폭력이 정당하냐는 질문에 대한 답은 아니기 때문입니다. 만약 포괄적으로 폭력이 정당하다고 한다면 한국의 군사체계는 물론이고 사법체계의 일부는 다시 구축되어야 할 것입니다. 만약 `어떤', 특수한 폭력이 정당하다고 한다면 정당하지 않은 폭력이 무엇인지 질문해야 할 것입니다. 얼마 안 있어 우리는 부당한 폭력과 정당한 폭력의 경계에 설 것입니다. 그때 다시 한번 잊었던 질문에 사로잡힐 것입니다. `어떤 폭력이 정당한가?' 라는 질문 말입니다.
    
    저는 폭력의 정당성을 판단하는 구조에 몸을 맡기는 대신 이 판단의 중심에 서서 조직화된 폭력에 합류하기를 거부할 것입니다. 또한 이 거부에 대해 요구되는 방식에 따라 책임질 것입니다.
    
    한국의 군대는 군인에게 마땅히 싸워야 할 적에 대해 교육합니다. 하지만 저는 그러한 교육에 저항합니다. 그 적이 이미 결정되어 있고, 그 결정을 전제로 한 명령을 받는다면 저는 그 명령에 대해 거부할 것입니다. 군대에서 생산되는 모든 명령은 적의 존재에 근거하고 있습니다.
    
    적이 누구인지는 제가 결정할 문제입니다. 그 누구도 이 신중한 결정체계에 끼어들어선 안 될 것입니다. 이에 대해서는 오롯이 제가 책임져야 할 것입니다. 이 말은 제가 국가가 결정한 주적의 개념에 대해, 실체에 대해, 제가 이를 받아들일 것인지에 대한 판단을 유보하겠다는 것입니다. 이 판단 유보에 대해서도 오롯이 제가 책임져야 할 것입니다.
    
    다시 한번 말씀드립니다. 본인, 김승해는 병역을 거부합니다. 이유는 군대에 가고 싶지 않기 때문입니다. 군인이 되고 싶지 않기 때문입니다. 군사의 경험으로 검증받아 이 나라에 합류하고 싶지 않기 때문입니다.
    \end{article}
    
    \begin{lyric}
    \hypertarget{uxbcf5uxb3c4-uxcc9cuxc0acuxb4e4}{%
    \chapter{복도, 천사들}\label{uxbcf5uxb3c4-uxcc9cuxc0acuxb4e4}}
    
    목매단 말들이 기어이 기어나가\\
    문간에 엎드렸다 잘린 머리통이 온 복도를\\
    굴러다녔다. 그 소리가 새벽을 소란스레 찢어 놓았다\\
    매달린 새벽에 우린 숨을 멈추었다\\
    오늘 새벽 우린 숨 없이 잠들었다 아침\\
    을 기약하지 않기로 하였다\\
    잘린 목에서 피거품이 일었다\\
    천사는 그 위를 걸어 지나가다 넘어지고 말았다\\
    잘린 머리통이 누운 천사의 어깨 위로 올라탔다\\
    천사가 진저리를 치며 일어났다
    
    우린 우려한다\\
    복도는 천사의 머리칼을 흔들 것이다\\
    흔들린 머리칼이 복도의 유리창을 세게 긁을 것이다\\
    우린 잘리다 만 귓불을 애써 붙이며\\
    목매단 해명을 늘어놓을 것이다\\
    목매단 말들이 기어이 기어나가 문간에 엎드렸다.
    \end{lyric}
    
    \begin{article}
    \hypertarget{uxc0b0uxc758-uxc911uxd131uxc5d0uxc11c}{%
    \chapter{산의 중턱에서}\label{uxc0b0uxc758-uxc911uxd131uxc5d0uxc11c}}
    
    산 중턱에 도착한 그들의 시야가 어두워지기 시작했다. 그들도 알지 못한 사이에 숲의 가장 깊은 곳에 도달한 것 같았다. 산의 비탈을 따라 내려오던 그들이었다. 그동안 희미하게나마 길을 밝히던 달빛도 감지되지 않았다. 하늘을 보려 고개를 올렸지만 달라 보이는 것은 아무것도 없었다. 가끔 그들의 눈앞을 지나가던 풀벌레들이 있었다. 달빛을 가로지르며 지나갈 때마다 수면이 깨어지듯 빛이 일었다. 지금은 아무것도 보이지 않았다. 곧 다시 밝아질 것이다. 이 모든 것들이 지나갈 것이다. 이런 낙관으로 당혹감을 떨쳐내야 했다. 몇 걸음 걸어 보았지만 곧 멈추었다. 한 발 내딛는 일이 마치 깊은 골짜기를 뛰어 건너는 일처럼 느껴졌다.
    
    아무것도 보이지 않았다. 그들은 온 세상이 어둠에 휩싸였다고 확신했다. 드디어 어둠이 그들을 찾아왔다고 생각했다. 이곳이 산의 중턱이었다.
    
    온 세상이 어둠에 휩싸이기 전이었다. 산에서 내려오다 마주친 중년 남성이 그들에게 말했다:
    
    \begin{quote}
    아무것도 보이지 않는 순간이 올 것이다. 산의 중턱에서 그 순간에 직면할 것이다. 다들 그때를 준비해야만 한다. 그때가 언제인지 알 수 없다. 산의 중턱이 어디서부터 시작할 것인지 알 수 없다. 산의 중턱이 어디서 끝나는지 알 수 없다. 나는 아무것도 모른다. 당신들에게 내가 말할 수 있는 단 한 가지 진실을 일러주겠다. 나는 이미 산의 중턱에 서 있다. 지금 이곳에서 빛을 기다리고 있다. 당신들은 아직 산의 중턱에 도달하지 않은 모양이다.
    \end{quote}
    
    그들이 다시 걷기로 한 것은 한참 뒤의 일이었다. 그들은 일어나 어떻게든 계곡 사이를 뛰어넘어 가기로 했다. 빛은 오지 않을 것이다. 이들의 걸음은 산의 중턱에 도착한 순간부터 도약의 연속이 되었다. 몇 번의 도약이 있고, 그들 중 하나가 넘어졌다. 팔꿈치가 쓰리지만 일어나 다시 걷기 시작했다. 둘 중 하나가 크게 다쳤고 얼마간 쉰 뒤 부축을 받으며 걸었다. 둘 중 하나가 넘어지고 앞니가 박살 났다. 턱이 두 동강 났다. 어깨가 빠졌다. 무릎이 쓸렸다. 발목이 꺾였다. 별수 없었다. 어둠 속에서 그들이 할 수 있는 일은 그럼에도 하던 일을 계속하는 것이었다. 멈추어 선 상태에선 아무런 가능성도 그들에게 찾아오지 않는다. 판단 유예도, 유보도, 추이를 살펴보는 모든 일들이 그들에겐 살점을 내어주는 일과 같았다. 그들은 이런 어둠을 준비해본 적도 없었고 아니, 그에 대한 준비가 가능하긴 한 건가. 다만 걷는 것 말고는 방법이 없어 보였다.
    
    아픔을 동여매며 바닥에 나뒹굴고 있던 그들을 한 사람이 지나쳐 갔다. 그는 바닥에 긴 불꽃을 그리며 지나가고 있었다. 불꽃에 드러난 것으로 보면 그는 마치 머리를 끌며 기어가는 것처럼 보였다. 그를 불러 어떻게 하면 불꽃을 그릴 수 있는지 물어보고자 했으나 그는 멈추지 않았다. 그가 지나간 자리에서 새빨간 흙이 꺼져가고 있었다. 그가 그린 궤적을 따라 내려오는 한 무리가 있었다. 무리 역시 모른 채 지나갔다. 붉은 흙이 사그라들었다. 그들은 다시 어둠 한가운데 걷는 신세가 되었다.
    
    그 불꽃의 정체를 알게 된 것은 그로부터 한참 지나서였다. 그들 중 하나가 쓰러지며 바닥에 머리를 심하게 다쳤는데 그 과정에서 두개골 일부가 드러나게 되었다. 넘어지지 않은 다른 하나가 두개골이 땅에 부딪히며 발생한 불꽃을 보게 되었다. 이렇게 불꽃의 정체가 규명되었다. 달빛도 별빛도 꺼진 이 세계에 두개골로 만든 빛만이 점멸하게 되었다. 그들은 애써 머리 가죽을 벗겨내고 땅에 머리를 부딪쳤다. 붉은 불꽃이 사그러지는 동안 서 있는 하나가 나아갈 길을 가늠하는 것이다.
    
    얼마 지나지 않아 그들은 역할을 분담하기로 했다. 어쨌든 이런 식으로 앞으로 가기 위해서는, 즉 빛을 만들기 위해서는 두개골이 드러나야만 했고, 그것은 당사자에게 무척 고통스러운 일이었다. 이 일을 한 사람만 하자는 것이다. 한 사람은 계속 머리를 끌고 한 사람은 계속 길을 찾아내는 것이다. 이제 한 사람에게 고통이 집중된다는 점에서 이전에 그들이 취한 방식보다 진보했다고 볼 수 있었다. 모두가 아플 수는 없는 일이었다. 머리를 끄는 동안 하나는 비명을 참기 위해 애써야만 했고, 나머지 하나는 억눌린 비명을 고스란히 들으며 그의 시야를 겹겹이 메운 의심을 몰아내야만 했다. 그것이 안쓰러워 나머지 하나가 두개골을 드러내고 땅을 기며 피고름에 시야를 가릴 순 없었다.
    
    이런 식의 여정이 이어지고, 놀라운 일이 발생하는 것을 관찰할 수 있었다. 머리를 끄는 쪽의 두개골이 발달하기 시작한 것이다. 이미 그의 머리에서 안구는 곪아 사라졌고, 그 사이를 응고된 핏덩어리가 가득 차 있었다. 코 또한 쓸려 사라졌다. 목뼈는 이미 뒤로 휘어 있었으며 비명을 감추기 위해 부푼 혀가 입안을 채우고 있었다. 바닥에 붙은 코 대신 터진 고막이 펄럭이는 그의 귀를 통해 호흡하기 시작했다. 몇 번이나 사지를 헤매고, 으깨어진 뇌가 귀에서 흘러나오는 바람에 빛이 빠르게 사그라진 것이 몇 번이었다. 그의 귀에는 굳은 뇌수의 거품이 엉겨붙어 있었다. 서로의 역할을 바꿀 수 없을 만큼 멀리 갔기에 아예 몹쓸 동정심으로 이 여정을 포기하기로 결정하려던 참이었다. 그런데 두개골이 빠른 속도로 발달하며 그 크기를 키우기 시작한 것이다. 이 덕분에 이전보다 더 밝은 빛을 낼 수 있었다.

\begin{center}
{\Large ...\par}
\end{center}

    \end{article}
    
    \begin{article}
    \hypertarget{uxb144-3uxc6d4-2uxc77c}{%
    \chapter{2013년 3월 2일}\label{uxb144-3uxc6d4-2uxc77c}}

\begin{center}
{\Large ...\par}
\end{center}
    
    이 책에 실릴 글들은 나의 수감 생활에 대해 아무것도 이야기해주지 않을 것이다. 한편 내가 늘 말해온 안정되고 안락했던 나의 수감 생활을 증언해줄 것이다. 안정과 안락함은 그때도, 지금도 내가 설정할 수 있는 최고의 목표니까 말이다.
    
    당신들의 안정을 빈다.
    \end{article}
    
    \backmatter
    
    \begin{lastnote}
    \begin{description}[itemsep=1pt,parsep=1pt]%
    \item[제목]%
    1x4 샘플북%
    \item[저자]%
    김승해
    \item[편집]%
    미루
    \item[디자인]%
    써드엔지니어링카르텔
    \item[출간일]%
    2018-10-12%
    \end{description}
    
    \begin{description}[itemsep=1pt,parsep=1pt]%
    \item[출판]%
    금치산자레시피
    \item[이메일]%
    gtsz.rcp@gmail.com
    \item[웹사이트]%
    http://gtszrcp.com
    \item[인스타그램]%
    gtsz.rcp
    \end{description}
    
    \begin{description}[itemsep=1pt,parsep=1pt]%
    \item[저작권]%
    이 책에 수록된 저작물 중 별도로 표기되지 않은 모든 저작물의 저작권은 저자에게 있습니다. 크리에이티브커먼즈 저작자표시-동일조건변경허락 4.0 국제 라이센스
    \end{description}
    \end{lastnote}
    \end{document}